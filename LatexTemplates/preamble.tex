%%%%%%%%%%%%%%%%%%%%%%%%%%%%%%%%%%
% My preamble file to define useful package
% importing and commands definition.
%
% Version 0.1.3
% Xiaming Chen - chenxm35@gmail.com
%
%% Math
\usepackage{amsmath}
\usepackage{amssymb}
\usepackage{array}
\usepackage{mathtools}
\usepackage{amsthm}
\usepackage{mathrsfs}

%% Algorithms
\usepackage{algorithmicx}
\usepackage{algorithm}
\usepackage{algpseudocode}

%% Tables
\usepackage{threeparttable,booktabs}
\usepackage{etoolbox}
\appto\TPTnoteSettings{\footnotesize}
\usepackage{multirow}
\usepackage[table]{xcolor}

%% Graphics and figures
\usepackage{graphicx}
\usepackage{epstopdf}
\usepackage{epsfig}
\usepackage{subfig}

%% Miscs
\usepackage{fixltx2e}
\usepackage{url}
% \usepackage{cite}
\usepackage{paralist}
\usepackage[numbers]{natbib}

%% Theorems
\theoremstyle{plain}% default
\newtheorem{thm}{Theorem}[section]
\newtheorem{lem}[thm]{Lemma}
\newtheorem{prop}[thm]{Proposition}
\newtheorem*{cor}{Corollary} % without numbering

%% Definition and example
\theoremstyle{definition}
\newtheorem{defnn}{Definition}
\newtheorem{defn}{Definition}[section]
\newtheorem{conj}{Conjecture}[section]
\newtheorem*{exmpn}{Example}
\newtheorem{exmp}{Example}[section]

%% Other remarks and notes
\theoremstyle{remark}
\newtheorem*{rem}{Remark}
\newtheorem*{note}{Note}
\newtheorem{case}{Case}

%% Definition equality
\newcommand{\defeq}{\vcentcolon=}
\newcommand{\eqdef}{=\vcentcolon}

%% Argument min/max
\newcommand{\argmin}{\mathop{\arg\min}}
\newcommand{\argmax}{\mathop{\arg\max}}

%% Bold emphasis
\newcommand{\emphbf}[1]{\emph{\textbf{#1}}}
\newcommand{\bftable}{\fontseries{b}\selectfont}

%% Figure files
\graphicspath{ {../figures/} }
\DeclareGraphicsExtensions{.eps,.pdf}

%% Set Letter paper size
%\setlength{\paperheight}{11in}
%\setlength{\paperwidth}{8.5in}
%\usepackage[pass]{geometry}

%%%%%%%%%%%%%%%%%%%%%%%%%%%%%%%%%%